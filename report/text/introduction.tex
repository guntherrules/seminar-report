\section{Introduction}
%this section should introduce the paper and ist goals.

Theory of Mind (ToM) is known in psychology as the ability to understand and reason about one's own and other individual's mental states such as differing beliefs or knowledge states \cite{theory_of_mind}. Most importantly, ToM is regarded to be a uniquely human ability \cite{tom_in_animals}. The research presented in the paper titled "Theory of Mind May Have Spontaneously Emerged in Large Language Models" attempts to contradict this understanding and suggests that large language models may already possess Theory of Mind while not being specifically designed to solve problems for which Theory of Mind is needed. By conducting three studies on the reasoning capabilities of large language models (LLM) such as GPT-3 and GPT-4 Kosinski attempts to prove this hypothesis and compares the results to research on Theory of Mind in developmental psychology, giving an interesting insight in the status of Artificial Intelligence compared to human intelligence.